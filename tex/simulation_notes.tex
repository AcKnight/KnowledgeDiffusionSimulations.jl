% !TEX program = pdflatex

\documentclass[11pt]{article}
\usepackage{amsmath,amsfonts,amsthm,amssymb,geometry,dsfont}
\usepackage[usenames,dvipsnames,svgnamesable]{xcolor}
\usepackage[capitalise,noabbrev]{cleveref} %
%\usepackage{natbib,url}
\crefname{equation}{}{} %
\crefname{assumption}{Assumption}{Assumptions}
\crefname{property}{Property}{Properties}
\geometry{left=1in,right=1in,top=0.6in,bottom=1in}
\newcommand{\set}[1]{\ensuremath{\left\{{#1}\right\}}}
\newcommand{\R}{\ensuremath{\mathbb{R}}}
\newcommand{\diff}{\ensuremath{\mathrm{diff}}}
\newcommand{\band}{\ensuremath{\mathrm{band}}}
\newcommand{\toep}{\ensuremath{\mathrm{toep}}}
\newcommand{\tridiag}{\ensuremath{\mathrm{tridiag}}}
\newcommand{\diag}{\ensuremath{\mathrm{diag}}}
\newcommand{\D}[1][]{\ensuremath{\partial_{#1}}}
\newcommand{\indicator}[1]{\ensuremath{\mathds{1}\left\{{#1}\right\}}}
\newcommand{\condexpec}[3][]{\ensuremath{\mathbb{E}_{#1}\left[{#2} \; \middle| \; {#3} \right]}}
\newcommand{\expec}[2][]{\ensuremath{\mathbb{E}_{{#1}}\left[ {#2} \right]}}
\geometry{left=1in,right=1in,top=0.6in,bottom=1in}
\newenvironment{psmallmatrix}
{\left(\begin{smallmatrix}}
	{\end{smallmatrix}\right)}

\theoremstyle{definition}
\newtheorem{example}{Examples}[section]

\bibliographystyle{ecta}
\begin{document}
\title{Notes on Discrete Simulations}
\author{Jess Benhabib, Jesse Perla, and Christopher Tonetti}
\maketitle

\section{Overview}
This package is 

\subsection{Linear Differential Equations}


Staley, Luttmer formulation, in log productivity,%
\[
\partial _{t}G\left( z,t\right) =\frac{\sigma ^{2}}{2}\partial
_{z}^{2}G\left( z,t\right) -\mu \partial _{z}G\left( z,t\right) -\alpha
G\left( z,t\right) \left( 1-G\left( z,t\right) \right) 
\]%
\[
G\left( t,-\infty \right) =0,\ \ G\left( t,+\infty \right) =1,\ \ \ G\left(
0,z\right) =G_{0} 
\]

Transform variables%
\[
\tilde{t}=\alpha t,\ \ \tilde{z}=\frac{\left( 2\alpha \right) ^{.5}}{\sigma }%
\left( z-\mu t\right) ,\ \ H\left( t,z\right) =1-G(t,z)
\]%
we get KPP:%
\begin{equation}
\partial _{t}H\left( t,z\right) =\partial _{z}^{2}H\left( t,z\right)
+H\left( t,z\right) \left( 1-H\left( t,z\right) \right)   \label{H}
\end{equation}%
with solution that depends on initial conditions given by $H\left(
0,z\right) $, and $v$ is th speed of the travelling wave solution to (\ref{H}%
).%
\[
H\left( t,z\right) =W_{v}\left( z-vt\right) 
\]%
For a solution we get a travelling wave, 
\[
\lim_{x\rightarrow \infty }W_{v}\left( x\right) \sim e^{-\gamma x}
\]%
where $\gamma $ is the smallest $\gamma $ that solves $v=\gamma +\gamma
^{-1}.$

If initial $H\left( 0,z\right) $ is Dirac or a compact perturbation of a
step function, then $v=2$, $\gamma =1,$ and there is a unique travelling
wave function.

If $H\left( 0,z\right) \sim e^{-\gamma x},\ 0<\gamma <1,\ $\ then $\gamma $
solves $v=\gamma +\gamma ^{-1}$ $>2$ as above. ALSO: From Murray,
Mathematical Biology, 1993, Springer, page 280, eq. (11.19): if $\gamma \geq
1,\ $for $H\left( 0,z\right) \sim e^{-\gamma x},$ then $v=2.$

Recovering the untransformed system: 

To reverse the transformation to get back to $G\left( t,z\right) $, we have: 
\begin{eqnarray}
\ G\left( t,z\right)  &=&1-H(\alpha t,\frac{\left( 2\alpha \right) ^{.5}}{%
\sigma }\left( z-\mu t\right) )\sim e^{-\gamma \tilde{z}}  \label{G1} \\
&=&e^{-\gamma \left( \frac{\left( 2\alpha \right) ^{.5}}{\sigma }\left(
z-\mu t\right) -v\alpha t\right) }=e^{-\gamma \frac{2\alpha }{\sigma }\left(
\left( z-\mu t\right) -v\frac{\sigma }{\left( 2\alpha \right) ^{.5}}\alpha
t\right) }=e^{-\gamma \frac{2\alpha }{\sigma }\left( z-\left( \mu +v\left( 
\frac{\alpha }{2}\right) ^{.5}\sigma \right) t\right) }  \label{G2} \\
&=&e^{-\gamma \frac{2\alpha }{\sigma }\left( z-\left( \mu +\left( \gamma +%
\frac{1}{\gamma }\right) \left( \frac{\alpha }{2}\right) ^{.5}\sigma \right)
t\right) }  \label{G3}
\end{eqnarray}%
for $z>\left( \mu +v\left( \frac{\alpha }{2}\right) ^{.5}\sigma \right) t$ $%
\ $\ along the travelling wave.Thus wave speed  for the original system $%
G\left( t,z\right) $ is $\left( \mu +\left( \gamma +\frac{1}{\gamma }\right)
\left( \frac{\alpha }{2}\right) ^{.5}\sigma \right) $ and the exponential
decay rate for log productivity is $\left( \gamma \frac{2\alpha }{\sigma }%
\right) ,$  unless $v=2$ and $\gamma =1$ because $H\left( 0,z\right) \sim
e^{-\gamma x}$ with $\gamma \geq 1,$ and then wave speed is $v=2$ so that $%
\left( \mu +2\left( \frac{\alpha }{2}\right) ^{.5}\sigma \right) =\mu
+\left( 2a\right) ^{.5}\sigma $ along the travelling wave for $z>\left( \mu
+\left( 2a\right) ^{.5}\sigma \right) t,$ and the decay rate of log
productivity distribution is $e^{-\frac{2\alpha }{\sigma }.}$

Note these cdf's for $G\left( t,z\right) $ are in log productivity: $log\
\left( z\right) \sim e^{-\gamma \frac{2\alpha }{\sigma }z},\ x\sim e^{\log
\left( z^{-\gamma \frac{2\alpha }{\sigma }}\right) }\sim z^{-\gamma \frac{%
2\alpha }{\sigma }}.$ For the pareto to have a mean we need $-\gamma \frac{%
2\alpha }{\sigma }.$

Mildred, in her SSRN paper had used the relations from above found in
McKean, Brunet-Derrida, Bramson, Murray and others, to study whether the
drop in recent growth is compatible with decreasing diffusion $\sigma ,$
which also decreases inequality as given by the tail index  $\frac{\gamma
\left( 2\alpha \right) ^{.5}}{\sigma },$ an inverse measure of inequality.
The tail index  however decreases with $\alpha $, compatible with decreasing
imitation  and rising inequality. One can argue therefore that decreasing $%
\alpha $ reflects increasing industry concentration, fostering inequality.
In Jones and Kim however higher growth  due to more leapfrogging and
inventions also generates  higher inequality, while we have recently
observed decreasing growth and higher inequality. Playing with both $\alpha $
and $\sigma $, Mildred could get both an increase in inequality and lower
growth, as in equation (\ref{G3}) above.


%\bibliography{simulation_notes}
\end{document}
